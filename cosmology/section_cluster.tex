\section{Cluster of Galaxies}
\subsection{Review}
\begin{itemize}
\item Cosmology with Clusters of Galaxies \citep[][bahcall00a, \#16 4/21/10]{bahcall00a}\\
In abstract, it states : "Rich clusters of galaxies, the largest
virialized systems known, place some of the most powerful constraints on
cosmology".  Questions to answer: 1) What is the mass density of the
universe?  2) How is the mass distributed? 
  \begin{itemize}
  \item Cluster Dynamics and M/L
     \begin{itemize}
     \item {\bf Velocity Dispersion} : motion of galaxies within clusters reflect
     the dynamical cluster mass within a given radius assuming the clusters
     are in hydrostatic equilibrium.
     \item {\bf Temperature of the hot intracluster gas} : traces the cluster mass.
     \item {\bf Weak Lensing} : distortion of background galaxies can be used
     to directly measure the intervening cluster mass.
     \end{itemize}
  \omegam $\simeq 0.2 \pm 0.1$ from the integration of over the entire observed 
  luminosity of the universe.
  \item Baryon Fraction\\ 
  The baryon fraction observed in clusters is :
  
  \item Cluster Abundance Evolution
  \end{itemize} 

\end{itemize}
\subsection{M/L}

