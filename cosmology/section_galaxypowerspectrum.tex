\section{Galaxy Power Spectrum}

\subsection{2dFGRS : 2 degree Field Galaxy Redshift Survey}
\begin{itemize}
\item Parameter constraints for flat cosmologies from cosmic microwave background and 2dFGRS power spectra \citep[][percival02a, \#217 4/21/10]{percival02a}\\
Joint analysis of the power spectrum from 2dFGRS and CMB.
CMB is COBE+BOOMERaNG, Maxima, DASI, VSA and CBI, this is before WMAP time.
\omegam h degeneracy is discussed and 2dFGRS tries to break the degeneracy.
\omegam $h^{3.4}$ = constant.

\item $\bigstar\bigstar\bigstar$ The Three-Dimensional Power Spectrum of Galaxies 
from the Sloan Digital Sky Survey 
\citep[][\#779, 5/3/10, tegmark04a]{tegmark04a}\\
The large-scale real-space power spectrum P(k) by using a sample of 205,443 galaxies from the Sloan Digital Sky Survey, covering 2417 effective square degrees with mean redshift z$\sim$0.1.
\omegam h=0.213$\pm$0.023, and $\sigma_{8}$=0.89$\pm$0.02 for $L^{*}$ galaxies, 
when fixing the baryon fraction \omegam / \omegab=0.17 and the Hubble parameter h=0.72.

\item The Three-dimensional Power Spectrum from Angular Clustering of Galaxies in 
Early Sloan Digital Sky Survey Data \citep[][dodelson02a, \#114 4/21/10]{dodelson02a}\\
$\Gamma=0.14 ^{+0.11} _{-0.06}$ (\omegam h)
\item The 2dF Galaxy Redshift Survey: the power spectrum and the matter content of the Universe \citep[][]{percival01a}\\
\omegam h = 0.20 $\pm$ 0.03 (note not $h^{2}$) and 
\omegam / \omegab =0.15 $\pm$ 0.07.

\end{itemize}
