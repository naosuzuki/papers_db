\section{Host Galaxy vs SNIa}
\subsection{Observation}
\begin{itemize}
\item $\bigstar$ Luminosity Functions of Type Ia Supernovae and Their Host Galaxies from the S
loan Digital Sky Survey \citep{yasuda10a} \\
Fig 12 : Luminosity (galaxy) vs. Number of SNe / galaxy.  Very interesting results on rates.  
SNe rates per galaxy is proportinal to the luminosity of galaxies for 3 orders of magnitudes.\\
Fig 17 : SN distance from host galaxy center vs. SN color (x1) : At the very center, there exist
some red color SNe but other than that, we don't see any clear trend.  For stretch (x1), we
don't see any trend in distance.

\item The Local Hosts of Type Ia Supernovae \citep{neill09a} \\

\item Supernovae in Early-Type Galaxies: Directly Connecting Age and Metallicity with Type Ia 
Luminosity
\citep[][\#32:4/30/10, gallagher08a]{gallagher08a} \\

\item $\bigstar\bigstar\bigstar$ Rates and Properties of Type Ia Supernovae as a Function 
of Mass and Star For mation in Their Host Galaxies 
\citep[][\#147:4/30/10,sullivan06a]{sullivan06a}\\
For the first time, correlation between stretch and SFR is shown clearly.
Passive galaxies with no star formation preferntially host faster declining/dimmer
SNeIa while brighter events are found in systems with ongoing star formation.

\item Chemistry and Star Formation in the Host Galaxies of Type Ia Supernovae
\citep[][\#67:4/30/10, gallagher05a]{gallagher05a} \\

\item $\bigstar\bigstar$ The Hubble diagram of type Ia supernovae as a function of host galaxy morphology \citep[][\#76, 4/30/10, sullivan03a]{sullivan03a} \\
\end{itemize}

\subsection{Theory}
