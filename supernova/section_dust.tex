\chapter{Dust Extinction}
\section{Dust Review}
\begin{itemize}
\item Interstellar Dust Grains \citep[][\#451, 5/18/10, draine03a]{draine03a}\\
\item Formation and destruction of dust grains \citep[][\#193, 5/20/10, salpeter77a]{salpeter77a}
\end{itemize}

\section{Dust Grain Studies}
\begin{itemize}
\item $\bigstar\bigstar$ Dust Evaporation in Protostellar Cores
\citep[][]{lenzuni95a} \\
Sublimation (phase transition from solid to gas) and chemical sputtering
of dust is studied.  According to their model, dust grains evapolates
at the temperatures of :
Carbon $\sim$ 1100K, Silicates $\sim$ 1300K, Aluminum Oxide $\sim$ 1720K.
\end{itemize}

\section{Dust in Circumsterllar Medium (CSM)}
\subsection{Dust Model}
\begin{itemize}
\item  Low R$_{V}$ from Circumstellar Dust around Supernovae \citep{goobar08a}\\
Ariel runs a Monte-Carlo simulation on the size of the dust particles with LMC
like dust
\item Dust around Type Ia Supernovae \citep[][\#39, 5/14/10, wang05b]{wang05b}\\
Lifan's Dust model paper.
\end{itemize}

\subsection{Time Variable Na DI line }
\begin{itemize}
\item A Second Case of Variable Na~I D Lines in a Highly Reddened Type Ia 
Supernova \citep{blondin09a}\\
SN2006X case is discussed in detail.
\item Variable Sodium Absorption in a Low-extinction Type Ia Supernova
\citep{simon09a}\\
Time variable Na ID line from SN2007le (Keck \& Hobby-Eberly Telescope) 
observation (-5 days to +90 days) is reported.
They note Ca II H \& K didn't move, thus CSM origin is suggested.
Unlike other SNe, the color is not too red.  E(B-V)=0.27 mag.
\item Circumstellar Shells in Absorption in Type Ia Supernovae
\citep{borkowski09a}\\
A theory paper to predict the variable Na ID line.
\item Detection of Circumstellar Material in a Normal Type Ia Supernova
\citep[][\#69, 6/2/10]{patat07a} \\
Time Variable Na~I line is discussed with ESO data on SN2006X
\end{itemize}

\subsection{CSM interacting SNe}
\begin{itemize}
\item Novae as a Mechanism for Producing Cavities around the Progenitors of SN 2
002ic and Other Type Ia Supernovae \citep{wood-vasey06a} \\
A model of recurrent nova projenitor forms CSM.
\item Nearby Supernova Factory Observations of SN 2005gj: Another Type Ia Supern
ova in a Massive Circumstellar Envelope
\citep{aldering06a}\\
\item SN 2003du: Signatures of the Circumstellar Environment in a Normal Type Ia Supernova?
\citep[][\#60, 6/2/10, gerardy04a]{gerardy04a}\\
SN2003du observation by Hobby-Eberly Telescope is reported.  High-Velocity CaII line,  
velocity of 18,000 km/s, is reported (CaII IR triplet around 8000A), and it is similar to
SN2000cx and SN2001el.
\end{itemize}

\subsection{Planetary Nebula}
\begin{itemize}
\item The PN.S Elliptical Galaxy Survey: Data Reduction, Planetary Nebula 
Catalog, and Basic Dynamics for NGC 3379 (M105)\citep{douglas07a}\\
The detection of 214 Planetary Nebulae is reported (191 in NGC3379 and 23
in NGC3384). NGC3379 is M105 and elliptical (S0) galaxy.

\item  Indications of a Large Fraction of Spectroscopic Binaries among Nuclei 
of Planetary Nebulae \citep[][\#48, 5/21/10, demarco04a]{demarco04a}\\
{\bf Abstract}:
Previous work indicates that about 10\% of planetary-nebula nuclei (PNNi) are photometrically variable short-period binaries with periods of hours to a few days. These systems have most likely descended from common-envelope (CE) interactions in initially much wider binaries. Population-synthesis studies suggest that these very close pairs could be the short-period tail of a much larger post-CE binary population with periods of up to a few months. We have initiated a radial-velocity (RV) survey of PNNi with the WIYN 3.5 m telescope and Hydra spectrograph, which is aimed at discovering these intermediate-period binaries. We present initial results showing that 10 out of 11 well-observed PNNi have variable RVs, suggesting that a significant binary population may be present. However, further observations are required because we have as yet been unable to fit our sparse measurements with definite orbital periods and because some of the RV variability might be due to variations in the stellar winds of some of our PNNi. 

\item Detection, Photometry, and Slitless Radial Velocities of 535 Planetary Nebulae 
in the Flattened Elliptical Galaxy NGC 4697 
\citep[][\#67, 5/25/10]{mendez01a} \\
535 PNe in an elliptical galaxy by VLT observation is reported.

\item Theoretical models of the planetary nebula populations in galaxies: The ISM oxygen abundance 
when star formation stops \citep{richer97a}\\
An attempt to model PNe in galaxies so that they can reproduce ISM [O/Fe]
chemical abundances.
\end{itemize}

\section{Dust in Interstellar Medium of the host galaxies I : Star Forming Galaxies}

\section{Dust in Interstellar Medium of the host galaxies II: Early Type Galaxies}
\begin{itemize}
\item $\bigstar\bigstar$ Galactic Winds \citep[][\#191, 5/18/10, veilleux05a]{veilleux05a}\\
Review of modern understanding of "Galactic Winds" from both observations and theory
\item $\bigstar$ Dust in the Cores of Early-Type Galaxies \citep[][\#184, 5/20/10, vandokkum95a]{vandokkum95a}\\
Using WFPC2 on HST, dust is detected in 31 out of 64 galaxies, 
although the sample is biased against the detection of dust.
\item $\bigstar\bigstar$ Galactic Wind \citep[][\#397, 5/20/10, mathews71a]{mathews71a}
The concept of "Galactic Wind" by supernovae is introduced
\end{itemize}

\section{Host Galaxy Dust (ISM) in general}
\begin{itemize}
\item Evolution of Dust Extinction and Supernova Cosmology
\citep[][\#25, 5/3/10, totani99a]{totani99a}\\

\item Do Type Ia supernovae prove $\Lambda >$0?
\cite[][]{rowanrobinson02a}\\
\end{itemize}

\section{Dust in Intracluster Medium}
\begin{itemize}
\item The kinematics of intracluster planetary nebulae and the on-going subclust
er merger in the Coma cluster core \citep{gerhard07a}\\
Studies of Intracluster Planetray Nebulae in the Coma Cluster. 37 PNe are identified
by Subaru (FOCAS).
\item Candidates for Intracluster Planetary Nebulae in the Virgo Cluster Based on the Suprime-Cam Narrow-Band Imaging in [O III] and H$\alpha$ \citep{okamura02a}\\
Intracluster 38 Planetary Nebulae candidates in Virgo Cluster are reported using Suprime-Cam
\end{itemize}

\section{IGM Dust}
