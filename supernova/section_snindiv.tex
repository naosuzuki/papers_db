
\section{Supernova Studies I : Normal Ia Supernova}
\subsection{\bf SN1981B}
\begin{itemize}
\item The Type I supernova 1981b in NGC 4536 - The first 100 days (branch83a)
\citet{branch83a}
\end{itemize}
\subsection{\bf SN1990N}
\begin{itemize}
\item Optical light curves of the Type IA supernovae SN 1990N and 1991T (lira98a)
\cite{lira98a}\\
{\bf (B-V)$_{Bmax}$=0.03:}
BVRI photometry is presented.  They report B-V at Bmax is 0.03.  Check: From Table
5, the closest data point to the Bmax is on JD 2,448,084.49 which is 1.8 days after
their estimated Bmax date (2,448,082.7). B-V=12.790-12.732=0.058.  MW extinction
is 0.026, thus the closest MW corrected B-V=0.032.
\end{itemize}


\subsection{\bf SN2002cd}
\begin{itemize}
\item CfA3: 185 Type Ia Supernova Light Curves from the CfA (hicken09a) 
\cite{hicken09a}\\
{\bf (B-V)$_{Bmax}$=0.464:}
Raw Data from the paper. z=0.0097 and K-correction is negligible.  
B maximum is around 2002-04-19 (MJD=52383.496), and magnitudes are:
B=17.145 and V=16.272.  Thus the raw B-V=17.145-16.272=0.873.
MW extinction is large: E(B-V)=0.409.  The extinction corrected
color is B-V=0.873-0.409=0.464.
\end{itemize}

\section{Supernova Studies II : 91bg-like}
\subsection{\bf SN1986G}
\begin{itemize}
\item The type 1a supernova 1986G in NGC 5128 - Optical photometry and spectra 
(phillips87a, \#147 : Mar 17, 2009) \citet{phillips87a} \\
{\bf (B-V)$_{Bmax}=$0.90$\pm$0.10 : } Early studies of SN~Ia: No modern parameterization.  
But they report B$_{max}$ date is on May 11 $\pm$ 1 (Table4).
They report B-V at Bmax is $0.90\pm0.10$ in Section III(E).
Check: From their photometry data (Table1), B-V on May 11.09 is 1.03$\pm$0.01.
Milky Way Extinction E(B-V)=0.115 from Schlegel Map.
Therefore, we expect B-V at the time of Bmax is 1.03-0.115=0.915.
SN1986G appeared in the
middle of dust lane in NGC5128 (Centaurus A).  Two interstellar absorption
line systems (CaII H\&K and NaI) are reported.  One of them is Milky Way ($v \sim 0 $ km/s), and the
other is associated with NGC5128 ($v \sim 430 $ km/s).
\end{itemize}
\subsection{\bf SN1989B}
\begin{itemize}
\item The Type IA supernova 1989B in NGC 3627 (M66) (wells94a, \#110 : Mar 17, 2009) 
\citet{wells94a}\\
{\bf (B-V)$_{Bmax}$=0.32 :} B$_{max}$ date is 2447565.3 $\pm$ 1.0 in Table 8.  
They report B$_{max}$=12.34$\pm$0.05 and V$_{Bmax}$=12.02$\pm$0.05 in Table 8, which
leads to (B-V)$_{Bmax}$=0.32.   They do not indicate Milky Way extinction correction
which is E(B-V)=0.032.  Thus 0.32-0.032=0.288.
Check: The closest data point is JD 2447566.8 and B-V is 12.41-12.02=0.39 from Table3.
However, there is another observation in Table 5 from photoelectric photometry.
On JD 2447465.81 B-V=12.29-11.83=0.36. With MW correction, it would be (B-V)$_{Bmax}$=0.328.
Supernova is in the middle of spiral galaxy's arm.
\end{itemize}
\subsection{\bf SN1991bg}
\begin{itemize}
\item The subluminous, spectroscopically peculiar type IA supernova 1991bg in the elliptical galaxy NGC 4374  (filippenko92b)
\cite{filippenko92b}\\
The original 91bg paper which defines fast decliner and its peculiarities.
This observation does not have B magnitude.

\item SN 1991bg - A type IA supernova with a difference (leibundgut93a) \cite{leibundgut93a}\\
{\bf (B-V)$_{Bmax}$=0.683:}
Observation with BVRI band magnitude no data before B$_{max}$.  \cite{krisciunas04a} below
estimates that the B$_{max}$ had happened 2 days before V$_{max}$ (Dec 14.7).
Check: The closest data point is Dec 13.50 which B-V=0.74.  One day after Dec 14.50 gives
B-V=0.85.  The color evolution is very fast.  Thus we can deduce (B-V)$_{Bmax}$ $<$ 0.74.
Milky Way extinction E(B-V)=0.041 so MW extinction corrected color would be (B-V)$_{Bmax}$ $<$ 0.7.
MLCS2k2 \cite{jha07a} estimates 0.683 and it is in a good agreement.

\item The properties of the peculiar type Ia supernova 1991bg. I. Analys
is and discussion of two years of observations (turatto96b) \citet{turatto96b}\\
{\bf (B-V)$_{Bmax}$=0.74:}
Both light curve (BVRI) and spectra are analyzed.  Good quality data but it still misses
Bmax data.  Bmax is on their JD 604 and the closest data point is +1.85 day and 
raw color is B-V=14.87-14.11=0.76 from Table 1.  With MW correction, 
(B-V)$_{Bmax}$=0.76-0.041=0.719.  They report the (B-V)$_{Bmax}$ $\sim$ 0.75 but not
clearly showed how they got this number.  However, they also report B$_{max}$-V$_{max}$=
14.75-13.96=0.79 from Table 3, and with MW correction, B$_{max}$-V$_{max}$=0.79-0.041
=0.749.  Thus reported number is a reasonable estimate and we adopt 0.74.  

\item Optical and Infrared Photometry of the Type Ia Supernovae 1991T, 1
991bg, 1999ek, 2001bt, 2001cn, 2001cz, and 2002bo (krisciunas04a) \cite{krisciunas04a}\\
\end{itemize}

\subsection{\bf SN1997cn}
\begin{itemize}
\item A New Faint Type IA Supernova: SN 1997CN in NGC 5490 (turatto98a) \citet{turatto98a}\\
{\bf (B-V)$_{Bmax}$=0.502 from MLCS2k2 fit:}
The data point misses the Bmax.  The date of Bmax is JD 2,450,588.  The closest data
is on 592.06 which is in phase +4.06 day and raw color B-V=17.46-16.62=0.82 from Table 1.
MW correction is E(B-V)=0.027, thus MW corrected color is B-V=0.82-0.027=0.793.
We expect the true (B-V)$_{Bmax}$ is bluer than this number.   They report B and V maximum
magnitude and B$_{max}$-V$_{max}$=17.2-16.55=0.65.  
With MW correction B$_{max}$-V$_{max}$=0.65-0.027=0.623.  Since at the time of Bmax,
V $>$ V$_{max}$, we expect (B-V)$_{Bmax}$ < 0.623.   MLCS2k2 fit by \citet{jha07a} gives
us 0.502.  For now, we adopt this number.
at Bmax should be less than 0.623.
\end{itemize}

\subsection{SN1998bp}
\begin{itemize}
\item UBVRI Light Curves of 44 Type Ia Supernovae (jha06) \citet{jha06}\\
{\bf (B-V)$_{Bmax}$=0.283: form LTCV data points} SALT2 fits find Bmax at MJD=50936.158.
On MJD=50937.398, data points are: Bmag=15.672$\pm$0.017, Vmag=15.313$\pm$0.010.
Thus, B-V=15.672-15.313=0.359.  Since MW extinction is E(B-V)=0.076, 
a(B-V)$_{Bmax}$=0.359-0.076=0.283.

\end{itemize}

\subsection{SN1999da}
\begin{itemize}
\item Optical and Infrared Photometry of the Type Ia Supernovae 1999da, 1999dk, 1999gp, 2000bk, and 2000ce
(krisciunas01a) \citet{krisciunas01a}\\
{\bf (B-V)$_{Bmax}$=0.611: form MLCS2k2}
Poorly organized paper.  SN1999da is presented but no Bmax date or color information is
provided.  From Table 6, the photometry data, the raw B-V color right before Vmax is
0.516 and it is getting redder so we can deduce (B-V)$_{Bmax}$ $>$ 0.516.  For now, we adopt
best fit color from MLCS2k2 which is 0.611.
However, the paper reports this is a faster decliner than a Type Ia SN characterized by $\Delta$=0.50
so it cannot be fitted with the MLCS vector. (Riess' original fit not Jha et al 2006)
\end{itemize}

\section{Supernova Studies III : 91T-like}
\subsection{\bf SN1991T}
\begin{itemize}
\item The peculiar Type IA SN 1991T - Detonation of a white dwarf? (filippenko92a)
\cite{filippenko92a}
This paper defines 91T-like supernova.  ``The absence of Si~II and Ca~II lines in SN1991T
reflects low abundances of intermediate-mass elements rather than an excitation effect.''
The paper discusses the continuum looks very similar to SN1990N.
\item Optical light curves of the Type IA supernovae SN 1990N and 1991T (lira98a)
\cite{lira98a}\\
{\bf (B-V)$_{Bmax}$=0.17:}
BVRI photometry is presented.  They report (B-V)$_{Bmax}$=0.17. In section 3.1,
they estimate Bmax is on JD 2,448,375.7$\pm$0.5.  In Table 6, they have 2 data points
on JD 2,448,375,65. B-V(1)=11.715-11.568=0.147, and B-V(2)=11.720-11.537=0.183, their
error bar is 0.016. MW extinction is 0.022 therefore, B-V(1)=0.125, and B-V(2)=0.161.
If we take the average, (B-V)$_{Bmax}$=0.143.
\end{itemize}

\section{Supernova Studies IV : Super Chandra}
\subsection{SN2009dc}
\begin{itemize}
\item Spectropolarimetry of Extremely Luminous Type Ia Supernova 2009dc:
 Nearly Spherical Explosion of Super-Chandrasekhar Mass White Dwarf
\citep{tanaka10a}\\
Studies spectropolarimetry of Super-Chandra SNIa from 5.6 to 89.5 days
after Bmax.
\end{itemize}

\subsection{SN2007if}
\begin{itemize}
\item Nearby Supernova Factory Observations of SN 2007if: First Total Mass Measu
rement of a Super-Chandrasekhar-Mass Progenitor \citep{scalzo10a}\\
\end{itemize}

\subsection{SN2006gz}
\begin{itemize}
\item The Luminous and Carbon-rich Supernova 2006gz: A Double Degenerate Merger?
\citep{hicken07a}\\
\end{itemize}

\subsection{SN2003fg : SNLS-03D3bb}
\begin{itemize}
\item $\bigstar\bigstar\bigstar$
The type Ia supernova SNLS-03D3bb from a super-Chandrasekhar-mass white dwarf star
\citep{howell06a} \\
The first discovey paper of super-Shandrasekhar mass SNIa.
\end{itemize}

\section{Supernova Studies V  : High Velocity Features (HVF)}
%\subsection{SN2002ic}

\section{Supernova Studies VI  : CSM interaction}

\subsection{SN2006X}
\begin{itemize}
\item The Detection of a Light Echo from the Type Ia Supernova 2006X in M100
\citep{wang08a}\\
\end{itemize}

\subsection{SN2005gj}
\begin{itemize}
\item Nearby Supernova Factory Observations of SN 2005gj: Another Type Ia Supern
ova in a Massive Circumstellar Envelope
\citep{aldering06a}\\
\end{itemize}

\subsection{SN2002ic : SNIa/SNIIn hybrid}
\begin{itemize}

\item $\bigstar\bigstar$ An asymptotic-giant-branch star in the progenitor 
      system of a type Ia supernova \citep[][\#137, 6/4/2010]{hamuy03a}\\

\item On the nature of the circumstellar medium of the remarkable Type Ia/IIn su
pernova SN 2002ic \citep[][\#39, 6/2/10, kotak04a]{kotak04a}\\
High-resolution, high S/N data is presented.  H$\alpha$ emission exibits P-Cygni
profile at the velocity of 100 km/s.  They conclude the projenitor is WD+Red Giant
and CSM interaction.

\item Subaru Spectroscopy of the Interacting Type Ia Supernova SN 2002ic: Eviden
ce of a Hydrogen-rich, Asymmetric Circumstellar Medium 
\citep[][\#47, 6/2/10]{deng04a}\\
Spectrum at 222 restframe days is presented. Strong H$\alph$ line, CaII lines and
strong interaction with CSM is suggested.

\item On the Hydrogen Emission from the Type Ia Supernova SN 2002ic
\citep[][\#50, 6/2/10, wang04b]{wang04b}\\
Lifan reports the detection of H$\alpha$ emission line from SN2002ic.
Spectropolarimetry data shows aspherically distributed hydrogen-rich CSM.

\item Novae as a Mechanism for Producing Cavities around the Progenitors of SN 2
002ic and Other Type Ia Supernovae \citep{wood-vasey06a} \\
A model of recurrent nova projenitor forms CSM.

\end{itemize}


\section{Supernova Studies VII : Peculiar Supernova}

\subsection{SN1999by}
\begin{itemize}
\item  The Luminosity of SN 1999by in NGC 2841 and the Nature of ``Peculiar'' Type Ia Supernovae
(garnavich04a) \citet{garnavich04a} \\
{\bf (B-V)$_{Bmax}$=0.434:} 
They report B$_{max}$-V$_{max}$=0.51$\pm$0.03, however, this is not what we are interested in.
Bmax JD is 1308.8$\pm$0.3 from Table 5, and the closest data point is -0.17 day.  The
raw color is B-V=13.66-13.21($\pm$0.02)=0.45.  MW extinction correction from Schlegel table
is E(B-V)=0.016.  Thus, MW corrected color is B-V=0.45-0.016=0.0434.
This supernova is peculiar because of its peak absolute magnitude, M$_{B}$=-17.15 which is
one of the least luminous events.  This paper reports the good correlation between
580nm depth.  This 580nm line was thought to be attributed solely to Si~II but in cooler
photospheres it is dominated by Ti~II.
\end{itemize}

\subsection{\bf SN2000cx}
\begin{itemize}
\item The Unique Type Ia Supernova 2000cx in NGC 524 (li01a) \citet{li01a}\\
{\bf (B-V)$_{Bmax}$=0.088:} 
No clear description of the color B-V at Bmax.
Bmax is on JD 2,451,752.2.  The closest data point is +0.76 day and raw
B-V=13.44-13.27=0.17.  MW correction E(B-V)=0.082.  Thus MW corrected 
B-V at Bmax is 0.17-0.082=0.088.  

SN2000cx is found in early type galaxy NGC524 whose Hubble type is S0.
The location is very far from nucleus so we expect to see minimum extinction
from the host galaxy.
Premaximum spectrum is similar to SN1991T (weak Si~II lines), but Si~II emerges
around maximum.   Asymmetric brightening is observed: Premaximum brightening is 
very fast but postmaximum decline is slow. 
Both the iron-peak and the intermediate-mass elements are found to be moving
at very high expansion velocities in the ejecta of SN2000cx.

\end{itemize}
\subsection{\bf SN2002cx}
\begin{itemize}
\item SN 2002cx: The Most Peculiar Known Type Ia Supernova (li03a) \citet{li03a}\\
{\bf (B-V)$_{Bmax}$=0.04$\pm$0.05:} 
In section 2.3, they report (B-V)$_{Bmax}$=0.04.
Check: Bmax is on JD 2,452,415.2, and the closest data points is on -2.4 days and
it gives us B-V(1)=17.77-17.73=0.04. The second closest is on +4.6 days and gives
us B-V(2)=17.85-17.45=0.4.  The color is evolving very quickly. MW extinction is
E(B-V)=0.032. B-V(1) would become 0.018, and we expect at Bmax it should be redder.
0.04 is a good estimate. 

They report that the premaximum spectrum is similar to 91T-like but the luminosity
is similar to 91bg-like.  Lines are dominated by Fe-group elements.

\item Late-Time Spectroscopy of SN 2002cx: The Prototype of a New Subclass of 
Type Ia Supernovae (jha06b) \citet{jha06b}\\
This paper presents late time spectra and shows very low expansion velocities
$\sim $ 700 $km/s$.  Low velocity O~I line is reported and it suggests unburned
materials.   The paper concludes that the sectral characteristics may be
consistent with pure deflagration models of Chandrasekhar-mass thermonuclear SNe.  
\end{itemize}

\subsection{\bf SN2003cg}
\begin{itemize}
\item Anomalous extinction behaviour towards the Type Ia SN 2003cg (eliasrosa06a)
\citet{eliasrosa06a}\\
{\bf (B-V)$_{Bmax}$=1.33$\pm$0.11:} 
Highly extinguished supernova.  They report interstellar line Na~ID and
CaII H\&K.
Contradicting results are reported in this paper:
They report E(B-V)=1.33 in abstract and table 11. Since they assume the color
of supernova at Bmax is 0.00 (section 3: they refer to Schaefer et al. 1995),
color excess translates into B-V color directly.  However, in the same section,
they refer to (B-V)$_{Bmax}$ is 1.08.  No clue how to add 0.25 mag. 
Also, on the Table 11, A$_{V}$ is reported 0.134.  MW Extinction form Schlegel
map is 0.031.  If R$_{B}$=4.1 then, 4.1 $\times$ 0.031 = 0.1271 and 0.0069 mag
off.  

Check: The closest data point to Bmax is +1.4 days.  B=15.97-0.016=15.954 
(-0.16 is s-correction) and V=14.72-0.011=14.709.  Thus B-V=1.245.
MW Extinction is E(B-V)=0.031.  B-V=1.245-0.031=1.215=1.22.  This number
agrees with their estimate within an error bar.  They claim they did 
simultaneous fit but 1.22 seems to be more consistent with the data.
\end{itemize}

\subsection{\bf SN2004dt}
\begin{itemize}
\item The early spectral evolution of SN 2004dt (altavilla07a) \citet{altavilla07a}\\
{\bf (B-V)$_{Bmax}$=-0.03$\pm$0.02:} 
This is a spectroscopy paper and they mention photometry paper is expected later.
However, they quote photometry data results.  From Table 1, (B-V)$_{Bmax}$ is -0.03.
(They assume normal supernova has (B-V)$_{Bmax}$=-0.07 which is SALT2? value.
Host galaxy is SBa type.  The highest degree of polarization is observed.
This supernova has a complex velocity structure and unburnt Oxygen is present.
The velocity is 16,700 km/s with some intermediate-mass elements (Mg,Si,Ca)
moving equally fast.
\end{itemize}

\subsection{\bf SN2005cf}
\begin{itemize}
\item ESC observations of SN 2005cf. II. Optical spectroscopy and the high-velocity features
(garavini07a) \citet{garavini07a}
\end{itemize}

\subsection{\bf SN2005hk}
\begin{itemize}
\item  The Evolution of the Peculiar Type Ia Supernova SN 2005hk over 400 Days (sahu08a)
\citet{sahu08a}\\
UBVRI photometry and spectroscopy is presented.  SN2005hk is similar to other
underluminous SNe~Ia, SN2002cx and SN2003gq.  The expansion velocity of the 
supernova is slower than normal Ia.  The best fit model suggests less energetic
deflagration explosion is the case.
\end{itemize}

\subsection{\bf SN2006gz}
\begin{itemize}
\item The Luminous and Carbon-rich Supernova 2006gz: A Double Degenerate Merger?
(hicken07a) \citet{hicken07a}\\
{\bf (B-V)$_{Bmax}$=+0.02:} (derived from photometry data points)
Super-Chandra candidate.  Unburned carbon signature in the spectrum, and one of
the slowest fading light ever seen in a Type Ia event.
They report host galaxy extinction corrected color B-V=-0.17 however, we wish to
know uncorrected raw color (but MW corrected).
Bmax date is JD 2,454,020.2.  From the raw photometry data taken from the cfa
web site, the closest data point is taken on 2454019.60500 (+0.6day).  
B=16.059$\pm$0.014, V=16.006$\pm$0.013.  Thus raw color is B-V=16.059-16.006=0.053.
MW extinction is E(B-V)=0.063.  (B-V)$_{Bmax}$=0.053-0.063=-0.01.
Just in case, let's try the second closest data point taken on 2454021.579 (+1 day).
B=16.064$\pm$0.015 and V=15.971$\pm$0.013, thus raw color is B-V=16.064-15.971=0.093.
MW extinction considered, (B-V)$_{Bmax}$=0.093-0.063=0.03.
In section 3.2, they mention absolute luminosities of $M_{B}=-19.17\pm0.04$ and
$M_{V}=-19.19\pm0.04$.  B-V (MW corrected) = $M_{B}-M_{V}=-19.17+19.19=+0.02$. This
number is consistent with the color we derived from the photometry data.




\end{itemize}
