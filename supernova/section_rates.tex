\section{SNIa Rates}
\subsection{SNIa Rates : Observation}
\begin{itemize}
\item $\bigstar$ The Type Ia Supernova Rate at z~0.5 from the Supernova Legacy Survey
\citep[][\#76:4/30/10,neill06a]{neill06a}\\
2 Years of CFHT SNLS data (73 spectroscopically confirmed SNeIa) to give SNIa rates
at $<z>$=0.47. Fig 11 component model is discussed. Prompt component (B), and delayed
component (A) is fitted to the observed rates.
\end{itemize}

\subsection{SNIa Rates : Theory/Model}
\begin{itemize}
\item The Role of Type Ia Supernovae in Chemical Evolution. I. Lifetime of Type Ia
Supernovae and Metallicity Effect \citep{kobayashi09a}\\
Chemical evolution is considered in terms of single degenerate scinario (SD).
The result reproduces metal distribution well.
\begin{itemize}
\item Fig 3a : Projenitor Mass vs. SNIa rates : the lower the mass, the higher the rates.
\item Fig 3b : Life Time of SNIa vs. SNIa rates : the shorter the life time, the higher the rates.
\item Fig 15 : z vs. SNIa rates for SD.  2 component model (Spirals and Ellipticals).
\item Fig 16 : z vs. SNIa rates for DD (double degenerate) .
\end{itemize}

\item $\bigstar\bigstar$ Two populations of progenitors for Type Ia supernovae? \citep[][\#152:4/30/10,mannucci06a]{mannucci06a}\\

\item $\bigstar\bigstar\bigstar$ The supernova rate per unit mass \citep[][\#204:4/30/10,mannucci05a]{mannucci05a}

\item $\bigstar\bigstar$ Low-Metallicity Inhibition of Type IA Supernovae and Galactic and Cosmic Chem
ical Evolution
\citep[][\#152, 5/4/10, kobayashi98a]{kobayaoshi98a}\\
The punch line is that SNeIa happens at the [Fe/H] $>$ -1.
This paper makes the following predictions:
 \begin{itemize}
 \item SNeIa are not found in the low iron abundance environment such as dwarf galaxies
 and outskirts of spirals
 \item SNeIa rates drops
 \end{itemize}

\end{itemize}

