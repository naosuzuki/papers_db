\section{Effect from Gravitational Lensing}
\begin{itemize}
\item A strategy for finding gravitationally lensed distant supernovae
\citep[][sullivan00a]{sullivan00a}\\
Gravitatoinally lensed high-z supernovae is discussed.
Predicts a high probability of seeing a supernova in a single
return visit with HST/ACS and NGST (James Webb).
The number of magnitude ampliflified SNe peaks at z=0.7 (Figure 3).
It also predicts high probability of finding 1-3 SNe in lensed cluster,
and as a demo, they report a SNe candidate discovery in AC114 at z=0.31
from HST archive data .

\item Reducing the gravitational lensing scatter of type Ia supernovae without introducing any extra bias
\citet[][\#2 4/26/10, jonsson09a]{jonsson09a}\\
"Bias arising from gravitational lensing correlations of individual SNeIa is negligible
for current and next generation surveys and the scatter from lensing can be reduced by approximately
a factor of 2."

\item Prospects and pitfalls of gravitational lensing in large supernova surveys
\citet[][\#6 4/26/10, jonsson08a]{jonsson08a}\\ 
Simulates gravitational lensing effect on the cosmological parameters.
With 500 SNe (SNLS) and outlier removal (2.5 $\sigma$) with BAO data (Case 2),
gravitational lensing would shift $-0.0052$ in \omegam and $-0.0051$ in
$w$ for a flat universe. 

\item  Weighing dark matter haloes with gravitationally lensed supernovae \citep{jonsson10a}
Attempts to measure the mass of galaxies from SNIa lensing from GOODS.

\end{itemize}
